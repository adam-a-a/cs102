\documentclass{article}
\usepackage{graphicx}
\usepackage[margin=1in]{geometry}
\usepackage{listings}
\usepackage{amsmath}

\begin{document}

\title{CS102: Week 2}

\maketitle

\section{Temperature Conversions}
\begin{enumerate}
\item Write a program that converts $100^{\circ}$ C to Fahrenheit. It should print out the following:
	\begin{quote}
	100 degrees Celsius is equivalent to \textit{F} degrees Fahrenheit
	\end{quote}
\textbf{Note:} replace \textit{F} with the converted temperature. 
\item Modify the program to convert arbitrary Celsius temperatures to Fahrenheit and change the printout accordingly.
\item Write a version of the program that converts Fahrenheit to Celsius. 
\end{enumerate}
Note: You may only submit this problem \textbf{once} for credit, either as part of recitation 2 or 3.

\section{Stadium Seating}
There are three catagories at a stadium. For a softball game:
\begin{itemize}
	\item Class A costs \$15
	\item Class B costs \$12
	\item Class C costs \$9
\end{itemize}
Write a program that asks how many tickets for each class of seats were sold, then displays the amount of income generated from ticket sales. Format your dollar amount in fixed-point notation, with two decimal places of precision, and be sure the decimal point is always displayed. 

\section{Average Rainfall}
Write a progra that calculates the average rainfall for three months. The program should ask the user to enter the name of each month (for example June or July) and the amount of rain in inches) that fell each month. The program should display a message similar to the following:
\begin{verbatim}
	The average rainfall for June, July, and August is 6.72 inches. 
\end{verbatim}
\break
\section{Pizza Pi}
Joe's Pizza Palace needs a program to calculate the number of slices a pizza of any size can be divided into. The program should perform the following steps:
\begin{enumerate}
	\item Ask the user for the diameter of the pizza in inches
	\item Calculate the number of slices that may be taken from a pizza of that size. 
	\item Display a message telling the number of slices.
\end{enumerate}

To calculate the number of slices that may be taken from the pizza, you must know the following facts:
\begin{itemize}
	\item Each slice should have an area of 14.125 inches
	\item The formular for the area of the whole pie is: Area=$\pi r^2$ 
\end{itemize}

\section{Average Temperature}
A thermometer is placed in various parts of the van and records the following temperatures: 99.9, 98.7, 100.3, 100.2, 99.5 
The average temperature in the van can be calculated as:
\begin{equation}
	van_{t} = \frac{1}{N}\sum_{i=1}^{i=N}t_{i}
\end{equation}
\begin{align*}
	\textbf{i} &= \text{current record}\\
	\textbf{N} &=  \text{total number of temperature records}
\end{align*}
Write a program that computes the average temperature and prints out: \\
\texttt{The average temperature in the van is} \textit{$van_{t}$}\\
\textbf{Hint:} Use an accumulator. 


\subsection{Extra Credit}
The error in the temperature reading can be estimated by calculating the variation in the temperature reading. The variation can be computed using the formula for standard deviation:
\begin{equation}
	var_{t} = \sqrt{\frac{1}{N}\sum_{i=1}^{i=N}(t_{i} - van_{t})^{2}}
\end{equation}
Write a program that computers the variation and prints out:\\
\texttt{The average temperature variation in the van records is} \textit{$var_{t}$}\\
\textbf{Note:} Replace $van_{t}$ and $var_{t}$ with the numbers you computed. 
\end{document}
