\documentclass{article}
\usepackage{graphicx}
\usepackage[margin=1in]{geometry}
\usepackage{listings}
\usepackage{amsmath}
\usepackage{minted}

\begin{document}

\title{CS102: File I/O}

\maketitle

\section{File Basics}
\subsection{Display}
Write a program to display the contents of a file. Your program should:
\begin{enumerate}
	\item ask the user for the name of a file and display an error if it doesn't exist.
	\item display the contents of that file or display an error if it is empty.
	\item close the file
\end{enumerate}

\subsection{Create}
Write a program that saves user input to a file. Your program should:
\begin{enumerate}
	\item ask the user for the name of a file and create the file if it doesn't exist
	\item ask the user for a number
	\item write that number to the file
	\item repeat steps 2 and 3 until the user enters -1
	\item close the file
\end{enumerate}

\subsection{Copy}
Write a function to copy the contents of a source file to a destination file. Your function should use the following prototype:
\begin{minted}{c++}
	void copy(string source, string destination);
\end{minted}
\begin{enumerate}
	\item open the source file for reading and display an error if it doesn't exist
	\item open the destination file for writing and display an error it it doesn't exist
	\item copy the contents from the source file to the destination file and display an error if the source is empty
	\item close both files
\end{enumerate}
Write a program to test your function. Your program should:
\begin{enumerate}
	\item ask the user for the name of the source file and display and error if it doesn't exist
	\item ask the user for the name of the destination file and create it if it doesn't exist
\end{enumerate}

\section{Data Wrangling}
\subsection{Save}
Modify the program in \textbf{Create}. Write function to store numbers in a vector
\begin{minted}{c++}
	void toVector(vector <int> vect);
\end{minted}
The function should:
\begin{enumerate}
	\item Ask the user how many numbers they want to enter
	\item Fill the array with user input
\end{enumerate}

 And then write another function to save the numbers to a file. Use the following protoypes:
\begin{minted}{c++}
	void save(string savefile, vector <int> vect);
\end{minted}
The function should:
\begin{enumerate}
	\item check for the savefile and create it if it doesn't exist
	\item write the numbers to the file
	\item close the file
\end{enumerate}
And write a program to test your functions. Your program should:
\begin{enumerate}
	\item call the toVector function
	\item ask the user if they'd like to save the numbers
	\item if the user says yes, ask for a file to save out to and save the numbers
\end{enumerate}

\subsection{Tabulate}
Write a program to read in a list of numbers, compute their sum, and save the sum out. First, write a function to read a list of numbers into an array:
\begin{minted}{c++}
	void file2Vector(string filename, vector <int> vect);
\end{minted}
The function should:
\begin{enumerate}
	\item ask the user for the name of the source file and display and error if it doesn't exist
	\item then read the numbers from the file into the vector
	\item close the input file
\end{enumerate}
Then write a function to compute and \textbf{return the sum} of the values in the array. Use the following prototype:
\begin{lstlisting}
	double sum(vector <int> vect);
\end{lstlisting}
And then write a function to save the sum to a file. Use the following prototype:
\begin{minted}{c++}
	void save(string savefile, int sum);
\end{minted}
The function should:
\begin{enumerate}
	\item check for the savefile and create it if it doesn't exist
	\item write the sum to the file
	\item close the file
\end{enumerate}
And then write the main that uses these functions. 
Then modify the program so that the sum is appended to the input file.

\section{Text Wrangling}
\section*{Character handling functions}
\begin{tabular}{|l|l|}
\hline
isalnum & Check if character is alphanumeric (function )\\ 
 \hline 
isalpha & Check if character is alphabetic (function )\\ 
 \hline 
isblank & Check if character is blank (function )\\ 
 \hline 
iscntrl & Check if character is a control character (function )\\ 
 \hline 
isdigit & Check if character is decimal digit (function )\\ 
 \hline 
isgraph & Check if character has graphical representation (function )\\ 
 \hline 
islower & Check if character is lowercase letter (function )\\ 
 \hline 
isprint & Check if character is printable (function )\\ 
 \hline 
ispunct & Check if character is a punctuation character (function )\\ 
 \hline 
isspace & Check if character is a white-space (function )\\ 
 \hline 
isupper & Check if character is uppercase letter (function )\\ 
 \hline 
isxdigit & Check if character is hexadecimal digit (function )\\ 
 \hline 
tolower & Convert uppercase letter to lowercase (function )\\ 
 \hline 
toupper & Convert lowercase letter to uppercase (function ) \\
\hline
\end{tabular}

\subsection{Counting Letters}
\begin{enumerate}
\item Write a function countC that returns the frequency of a character in a string. The function should be case sensitive. Use the following prototype:
\begin{minted}{c++}
int countC(string line, char c);
\end{minted}
For example, if it was called in the following way:
\begin{minted}{c++}
countC("Hello", "o");
\end{minted}
countC should return \textbf{1}.
\item print out the frequency of all the letters in the alphabet in a given string (case sensitive)
\item  print out the frequency of all the letters in the alphabet in a given string (case insensitive)
\item print out the frequency of punctuation
\item modify the program to print out the frequency of all letters in a file
\item \textbf{Extra Credit}: Write a function which prints out which letter is most frequent and which letter is least frequent (in an arbitrary file);
\end{enumerate}

\subsection{Counting Characters}
\begin{enumerate}
\item Write a function to count the number of characters (in total) in a given file. 
\item Write a function to print out every 5th character
item Write a function to print out a string in chunks of 5. For example given the string "Today is thursday.", your function will print out:
\begin{minted}{bash}
Today
 is t
hursd
ay.
\end{minted}
\item Modify the function to strip out spaces before it chunks
\item Modify the function to print out arbitrarily sized (3,4, 7? character) chunks
\item Modify the program to work with files.
\end{enumerate}

\end{document}