\documentclass[addpoints,12pt]{exam}
\usepackage[margin=1in]{geometry}
\usepackage{listings}
\usepackage{multicol}
\usepackage{tabularx}
\newcounter{matchleft}
\newcounter{matchright}

\newenvironment{matchtabular}{%
  \setcounter{matchleft}{0}%
  \setcounter{matchright}{0}%
  \tabularx{\textwidth}{%
    >{\leavevmode\hbox to 1.5em{\stepcounter{matchleft}\arabic{matchleft}.}}X%
    >{\leavevmode\hbox to 1.5em{\stepcounter{matchright}\alph{matchright})}}X%
    }%
}{\endtabularx}

\begin{document}
\header{CS102}{Quiz 6}{}

\begin{center}
\fbox{\fbox{\parbox{5.5in}{\centering
Answer the questions in the spaces provided on the
question sheets. If you run out of room for an answer,
continue on the back of the page.\\
\textbf{Show all work}. \\
Credit \textbf{will not} be given if work is not shown.}}}
\end{center}
\vspace{0.1in}
\makebox[\textwidth]{Name and section:\enspace\hrulefill}
\begin{center}
\gradetable[h][questions]
\end{center}

\begin{questions}
\question [5] Given the following code:
\begin{multicols}{2}
\begin{lstlisting}
#include <iostream>
using namespace std;

void f1(int var1, int var2);
void f2(int &var1, int &var2);

int main(){
	int a = 1;
	int b = 2;
	//1.
	f1(a, b);
	//2. 
	f2(a, b);
	//3. 
	f1(a, b);
	//4. 
	f2(a, b);
	//5.
	return 0;
}

void f1(int var1,  int var2){
	var1 = 2;
	var2 = 3;
}

void f2(int &var1, int &var2){
	var1 = 2;
	var2 = 3;
}

\end{lstlisting}
\columnbreak
Fill in the value of a and b at:
	\begin{enumerate}
		\item a = \underline{\hspace{1cm}} b = \underline{\hspace{1cm}}    
		\item a = \underline{\hspace{1cm}} b = \underline{\hspace{1cm}}
		\item a = \underline{\hspace{1cm}} b = \underline{\hspace{1cm}}   
		\item a = \underline{\hspace{1cm}} b = \underline{\hspace{1cm}}   
		\item a = \underline{\hspace{1cm}} b = \underline{\hspace{1cm}}   
	\end{enumerate}   
\end{multicols}
\question [5]
Given the following prototype, write a function that \textbf{returns} the hypotenuse. You're allowed to use the cmath library.
\begin{lstlisting}
	double hyp(double a, double b);
\end{lstlisting}
 \textbf{Extra credit:} Write a program to test your function. 
\end{questions}



\end{document}
