\documentclass{article}
\usepackage{graphicx}
\usepackage[margin=1in]{geometry}
\usepackage{amsmath}
\usepackage{minted}

\begin{document}

\title{CS102: Week 4}

\maketitle

\section*{Even and Odd}
Write, run, and test a C++ program that accepts a user-input integer number and determines whether it’s even or odd. Display the entered number and the message \textbf{Even} or \textbf{Odd}.

\section*{Fizz Buzz}
Write, run, and test a C++ program that accepts a user-input integer number and  
\begin{itemize}
	\item print \textbf{Fizz} for multiples of 3
	\item print \textbf{Buzz} for multiples of 5
	\item print \textbf{FizzBuzz} for multiples of 3 and 5
\end{itemize}

\section*{Stop Light}

Write, run, and test a C++ program that accepts a user-input character representing the current light of the stop light. Determine and the print out what the next light will be. Use a switch statement to complete this task. The valid inputs are: R, G, Y.
\begin{verbatim}
	What is the current light?  R
	The next light is G.
\end{verbatim}

Modify the program so that the following inputs are valid: R, r, G, g, Y, y
\begin{verbatim}
	What is the current light?  r
	The next light is G.
\end{verbatim}


\section*{Minimum of Three Numbers}
Find the minimum of three numbers. Your program has the following constraints:
\begin{itemize}
	\item inputs are unsorted
	\item only allowed to use pairwise comparisons
		\begin{description}
			\item [no] 
\begin{minted}{c++}
(a>b) && (b>c)
\end{minted}
\item[yes] 
\begin{minted}{c++} 
if (a>b) { 
     if (b>c){...}
...}
\end{minted}
\end{description} 
\end{itemize}


\section*{Supplement: Boolean Algebra Evaluator}
Write, run, and test a C++ program that evaluates a 2 value boolean algebra equation, such as A and B.  The program accepts the truth values of A and B and the operation and returns the truth value of the expression. 
\begin{verbatim}
	What is the truth value (0,1) of A? 0
	What is the truth value (0,1) of B? 1
	What is the operation (and, or)? and
	The truth value of 0 and 1 is 0
\end{verbatim}

\section*{Median of Three Numbers}
Find the median of three numbers. Your program has the following constraints:
\begin{itemize}
	\item inputs are unsorted
	\item only allowed to use pairwise comparisons
		\begin{description}
			\item [no] 
\begin{minted}{c++}
(a>b) && (b>c)
\end{minted}
\item[yes] 
\begin{minted}{c++} 
if (a>b) { 
     if (b>c){...}
...}
\end{minted}
\end{description} 
\end{itemize}

\end{document}
