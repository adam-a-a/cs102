\documentclass{article}
\usepackage{graphicx}
\usepackage[margin=1in]{geometry}
\usepackage{minted}
\usepackage{amsmath}

\begin{document}

\title{CS102: Week 2}

\maketitle
\section*{Address Book}
\begin{enumerate}
\item Initialize the following variables with your information:
	\begin{itemize}	
 		\item firstname
		\item lastname
		\item streetaddress
		\item city
		\item state
		\item zipcode
	\end{itemize}
And then print out the following using the values you've set
\begin{verbatim}
<firstname> <lastname>
<streetaddress> 
<city>, <state>, <zipcode>
\end{verbatim}
 \item Repeat the above, but this time ask the user to input the information
\end{enumerate}

\section*{Temperature Conversions}
The formula to convert Celsius to Fahrenheit is:
\begin{equation}	
^{\circ} F =  \frac{9}{5} {^{\circ}C} + 32
\end{equation}
\\
\begin{enumerate}
\item Write a program that converts $100^{\circ}$ C to Fahrenheit. It should print out the following:
	\begin{quote}
	100 degrees Celsius is equivalent to \textit{F} degrees Fahrenheit
	\end{quote}
\textbf{Note:} replace \textit{F} with the converted temperature. 
\item Modify the program to convert arbitrary Celsius temperatures to Fahrenheit and change the printout accordingly.
\item Write a version of the program that converts Fahrenheit to Celsius. 
\end{enumerate}

\section*{Pointers and References}
Write a C++ program that includes the following code:
\begin{minted}{c++}
int b;
int& a = b;
a = 10;
cout<<"a="<<a<<" b="<<b<<endl;

int d;
int *c = &d;
*c = 10;
cout<<"c="<<c<<" d="<<d<<endl;
\end{minted}
Explain what each line does. Then modify the program so that a and c are set using user input (hint: substitute a variable for 10); 

\section*{Extra: Memory Allocation}
Using the \texttt{sizeof()} operator determine the number of bytes your computer uses to store:
\begin{enumerate}
	\item an integer, a character, and a double-precision number
	\item the \textbf{address} of an integer, a character, and a double-precision number
\end{enumerate}
Hint: \texttt{sizeof(*int)} can be used to determine the number of memory bytes used for a pointer to an integer.) Would you expect the size of each address to be the same? Why or why not?

\end{document}