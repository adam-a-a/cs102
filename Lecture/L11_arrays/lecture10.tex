\documentclass[xcolor={dvipsnames}]{beamer}
\usepackage{amsmath,amsfonts,amssymb,pxfonts,eulervm,xspace}
\usepackage{graphicx}
\usepackage{multimedia}
\usepackage{media9}
\usepackage{minted}
\usepackage{mathtools}
\usepackage{animate}
\usepackage{tabularx}

\graphicspath{{./figures/}}
\usetheme{ccnycrest}


\begin{document}

\title{ CS102: Arrays and Vectors}
\author{Hannah Aizenman}


\begin{frame}
	\titlepage
\end{frame}


\begin{frame}[fragile]{Array}
	\begin{block}{Array named "numbers"}
	\begin{table}
	\Huge
	\begin{tabular}{|c|c|c|c|c|}
	\hline
	2  & 4 & 6 & 8 & 10\\
	\hline
	\end{tabular}
	\end{table}
	\end{block}
	\pause
	\begin{block}{C++ implementation}
	\begin{minted}{c++}
	int numbers[5] = {2, 4, 6, 8, 10};
	\end{minted}
	\end{block}
\end{frame}

\begin{frame}[fragile]{Generic form of specifying arrays}
\begin{center}
	arrays can contain values of any type
\end{center}
\begin{block}{Declaration}
	\begin{minted}{c++}
	type name[length];
	\end{minted}
\end{block}
\begin{block}{Assignment}
	\begin{minted}{c++}
	type name[length] = {sequence of values};
	\end{minted}
	\end{block}
\end{frame}

\begin{frame}[fragile]{Sample Arrays}
	\begin{center}
		arrays can contain values of any type
	\end{center}
	\begin{minted}{c++}
		int numbers[5] = {2, 4, 6, 8, 10};
		double grades[3] = {3.5, 2.3, 1.0};
		char hello[6] = {'h', 'e', 'l', 'l', 'o', '\0'};
		bool mask[7] = {1, 1, 0, 0, 1, 1, 0};
	\end{minted}
\end{frame}



\begin{frame}{Array Indexing}
	\begin{block}{int numbers[5]}
	\begin{table}
	\Huge
	\begin{tabular}{|c|c|c|c|c|c|}
	\hline
	value & 2  & 4 & 6 & 8 & 10\\
	\hline
	{\color{red} index} &  {\color{red} 0}   &  {\color{red} 1}  &   {\color{red} 2}  &  {\color{red} 3}  &  {\color{red} 4} \\
	\hline
	\end{tabular}
	\end{table}
	\end{block}
	\pause
	\begin{block}{For element e in array numbers}
		\begin{description}
			\item[value] \textbf{what} the element is
			\item[index] \textbf{where} the element is 
		\end{description}
	\end{block}
\end{frame}

\begin{frame}[fragile]{Manipulating values using indexing}
	\begin{block}{General form}
	\begin{minted}{c++}
		name[index]
		name[index] = value;
	\end{minted}
	\end{block}
	\pause
	\begin{block}{Printing out the value at index 3}
	\begin{minted}{c++}
		cout<<numbers[3]<<endl;
	\end{minted}
	\end{block}
	\pause
	\begin{block}{Assign a value to index 4}
	\begin{minted}{c++}
		numbers[4] = 42;
	\end{minted}
	\end{block}
\pause
	\begin{block}{Input an element to index 2}
	\begin{minted}{c++}
		cin>>numbers[2];
	\end{minted}
	\end{block}
\end{frame}

\begin{frame}[fragile]{2D Array}
	\begin{block}{Array named "numbers"}
	\begin{table}
	\Huge
	\begin{tabular}{|c|c|c|c|c|}
	\hline
	2  & 4 & 6 & 8 & 10\\
	\hline
	3  & 5 & 7 & 9 & 11\\
	\hline
	\end{tabular}
	\end{table}
	\end{block}
	\pause
	\begin{block}{C++ implementation}
	\begin{minted}{c++}
	int numbers[2][5] = {{2, 4, 6, 8, 10},
	                     {3, 5, 7, 9, 11}};
	\end{minted}
	\end{block}
\end{frame}

\begin{frame}{2D Array Indexing}
	\begin{block}{int numbers[2][5]}
	\begin{table}
	\Huge
	\begin{tabular}{|c|c|c|c|c|c|}
	\hline
	{\color{red}index} &  {\color{red} 0}   &  {\color{red} 1}  &   {\color{red} 2}  &    {\color{red} 3}  &  {\color{red} 4} \\

	\hline
	 {\color{red} 0} & 2  & 4 & 6 & 8 & 10\\
	\hline
	 {\color{red} 1}  & 3  & 5 & 7 & 9 & 11\\
	\hline
	\end{tabular}
	\end{table}
	\end{block}
	\pause
	\begin{block}{For element e in array numbers}
		\begin{center}
		row index then column index
		\end{center}
	\end{block}
\end{frame}

\begin{frame}[fragile]{Manipulating values using indexing}
	\begin{block}{General form}
	\begin{minted}{c++}
		name[row][column]
		name[row][column] = value;
	\end{minted}
	\end{block}
	\pause
	\begin{block}{Printing out the value at row index 0 and column index 3}
	\begin{minted}{c++}
		cout<<numbers[0][3]<<endl;
	\end{minted}
	\end{block}
	\pause
	\begin{block}{Assign a value to row index 1 and column index 4}
	\begin{minted}{c++}
		numbers[1][4] = 42;
	\end{minted}
	\end{block}
	\pause
	\begin{block}{Input an element to row index 0 and column index 2}
	\begin{minted}{c++}
		cin>>numbers[0][2];
	\end{minted}
	\end{block}
\end{frame}

\begin{frame}[fragile]{Pointer Based Indexing}
\begin{block}{int arr[5] = {2,4,6,8,10};}
	\begin{table}
	\Huge
	\begin{tabular}{|c|c|c|c|c|c|}
	\hline
	value & 2  & 4 & 6 & 8 & 10\\
	\hline
	\end{tabular}
\end{table}
\end{block}
\begin{block}{Access a value}
	\begin{table}
	\begin{tabular}{|c|c|c|c|c|c|}
	\hline
	arr[index]  & arr[0]  & arr[1] & arr[2] & arr[3] & arr[4]\\
	\hline
        *(arr+index) & *arr & *(arr+1) & *(arr+2) & *(arr+3) & *(arr+4)\\
	\hline
	\end{tabular}
\end{table}
\begin{description}
	\item[array access: arr[i]] the ith value 
	\item[pointer dereference: *(arr+i)] the value i spaces over from arr
\end{description}
\end{block}
\end{frame}

\begin{frame}[fragile]{Pointer Based Arithmetic}
\begin{block}{}
	\begin{minted}{c++}
int arr[5] = {2,4,6,8,10};
int *ptNum = &arr[2];
//ptNum points to the address of the 2nd element in arr
//this element has the value 6
	\end{minted}
\end{block}

\begin{tabular}{|l|l|l|}
\hline
ptNum++ & use the pointer & ptNum points to 6\\
                   & then increment it  & then moves to 8 \\
\hline
++ptNum & increment the pointer & ptNum moves to 10\\
		   & then use it & then points to 10\\
\hline
ptNum\text{-}\text{-} & use the pointer & ptNum points to 10\\
				 & then decrement it & then moves to 8\\
\hline
\text{-}\text{-}ptNum & decrement the pointer & ptNum moves to 6\\
		& then use it & then points to 6\\
\hline
\end{tabular}
\center
\alert{incrementing and decrementing pointers shifts \textbf{Addresses}}
\end{frame}

\begin{frame}[fragile]{Generic: Arrays as a function argument}
	\begin{block}{prototype}
	\begin{minted}{c++}
		function_type function_name(array_type array_name[]);
	\end{minted}
	\end{block}
	\pause
	\begin{block}{implementation}
	\begin{minted}{c++}
		function_type function_name(array_type array_name[]){
			//function body
		}
	\end{minted}
	\end{block}
\end{frame}

\begin{frame}[fragile]{Arrays as a function argument}
	\begin{block}{prototype}
	\begin{minted}{c++}
		int max(int arr[], const int LENGTH);
	\end{minted}
	\end{block}
	\pause
	\begin{block}{implementaton}
	\begin{minted}{c++}
int max(int arr[], const int LENGTH){
    int maxElement = arr[0];
    for (int i=0; i<LENGTH; i++){
        if (arr[i]>maxElement){
            maxElement=arr[i];
        }
    }
    return maxElement; 
}
	\end{minted}
	\end{block}
\end{frame}

\begin{frame}[fragile]{Generic: 2D Arrays as a function argument}
	\begin{block}{prototype}
	\begin{minted}{c++}
		function_type function_name(array_type array_name[][]);
	\end{minted}
	\end{block}
	\pause
	\begin{block}{implementation}
	\begin{minted}{c++}
		function_type function_name(array_type array_name[][]){
			//function body
		}
	\end{minted}
	\end{block}
\end{frame}

\begin{frame}[fragile]{Arrays as a function argument}
	\begin{block}{prototype}
	\begin{minted}{c++}
		int max(int arr[][], const int ROWS, const int COLUMNS);
	\end{minted}
	\end{block}
	\pause
	\begin{block}{implementaton}
	\begin{minted}{c++}
int max(int arr[][], const int ROWS, const int COLUMNS){
    int maxElement = arr[0][0];
    for(int i=0; i<ROWS; i++){
        for(int j=0; j<COLUMNS; j++){
            if (arr[i][j]>maxElement){
                maxElement=arr[i][j];
            }
        }
    }
    return maxElement;
}
	\end{minted}
	\end{block}
\end{frame}

\begin{frame}{Vectors}

\begin{block}{Similar to an array}
	\begin{itemize}
		\item holds a sequence of elements
		\item stores elements in adjoining memory locations
		\item supports the array subscript operator \textbf{\[\]}
	\end{itemize}
\end{block}

\begin{block}{Advantages over an array}
	\begin{itemize}
		\item do not need to know how many elements it will contain
		\item automatically increases in size to accommodate new values
		\item has functions for reporting number of elements
		\item can easily set a default value
	\end{itemize}
\end{block}
\end{frame}

\begin{frame}[fragile]{Using Vectors}
\begin{minted}{c++}
#include <iostream>
#include <vector> //include vector lib

using namespace std;

int main()
{
   //create a vector containing 10 elements
   vector <int> numbers(10);
   //set 1st element to 1
   numbers[0] = 1;
   cout<<numbers[0]<<endl;
   //set 2nd element
   numbers[1] = 2;
   cout<<numbers[1]<<endl;
}
\end{minted}
\end{frame}

\begin{frame}[fragile]{Using Vectors: Default Value}
\begin{minted}{c++}
#include <iostream>
#include <vector> //include vector lib

using namespace std;

int main()
{
   //create an array of size 10 
   //with all values set to 2
   vector <int> numbers(10,2);
   cout<<numbers[1]<<endl;
   //change the value at index i
   numbers[1] = 4;
   cout<<numbers[1]<<endl;
}
\end{minted}
\end{frame}
\begin{frame}[fragile]{Using Vectors: Unknown Size}
\begin{minted}{c++}
#include <iostream>
#include <vector> //include vector lib

using namespace std;

int main()
{
   vector <int> numbers;
   //use push_back to add 1st value
   numbers.push_back(1);
   cout<<numbers[0]<<endl;
   //use push_back to add 2nd value
   numbers.push_back(2);
   cout<<numbers[1]<<endl;
}
\end{minted}
\end{frame}

\begin{frame}[fragile]{Creating (defining) a Vector}
\begin{block}{Empty Vector: \texttt{vector <dtype> name;}}
\begin{minted}{c++}
vector <string> students;
\end{minted}
\end{block}

\begin{block}{Empty, Initial Length: \texttt{vector <dtype> name(len);}}
\begin{minted}{c++}
vector <string> students(30);
\end{minted}
\end{block}

\begin{block}{Length, Defaults: \texttt{vector <dtype> name(len,val);}}
\begin{minted}{c++}
vector <string> students(30, "Anonamouse");
\end{minted}
\end{block}

\begin{block}{Copy Vector: \texttt{vector <dtype> name1(name);}}
\begin{minted}{c++}
vector <string> new_students(students);
\end{minted}
\end{block}
\end{frame}

\begin{frame}{Common Vector Operations}
\begin{tabularx}{\textwidth}{|l|X|}
\hline
\texttt{name[i]} & value at location i, doesn't check if i is valid\\
\hline
\texttt{name.at(i)} & value at location i, checks if i is a valid index\\
\hline
\texttt{name.push\_back(value)} & adds value to end of vector\\
\hline
\hline
\texttt{name.size()} & number of elements in the vector\\
\hline
\texttt{name.empty()} & boolean-returns true of the vector is empty\\
\hline
\texttt{name.capacity()} & maximum number of elements\\
\hline
\hline
\texttt{name.clear()} & clears a vector of all its elements\\
\hline
\texttt{name.pop\_back()} & removes the last element in the vector\\
\hline
\hline
\texttt{name.reverse()} & reverses all the elements in the vector\\
\hline
\texttt{name.swap(name2)} & swaps the elements in name with the elements in name2\\
\hline
\end{tabularx}
\end{frame}
\end{document}

