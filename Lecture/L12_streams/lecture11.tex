\documentclass[xcolor={dvipsnames}]{beamer}
\usepackage{amsmath,amsfonts,amssymb,pxfonts,eulervm,xspace}
\usepackage{graphicx}
 \usepackage{multimedia}
\usepackage{media9}
\usepackage{minted}
\usepackage{mathtools}

\usepackage{animate}

\graphicspath{{./figures/}}
\usetheme{ccnycrest}


\begin{document}

\title{ CS102: I/O Streams}
\author{Hannah Aizenman}



\begin{frame}
	\titlepage
\end{frame}

\begin{frame}{Input Stream}
	\begin{figure}
		\includegraphics[width=1\textwidth]{read}
	\end{figure}
\end{frame}

\begin{frame}[fragile]{Opening and closing an input file}
\begin{minted}{c++}	
#include <iostream>
#include <fstream>
#include <string>

int main(){
        std::string filename = "data.txt";
        std::ifstream inFile;
        inFile.open(filename.c_str());
        if (!inFile){
                //error handling
        }else{
                //process file
        }
        inFile.close();
        return 0;
}
\end{minted}
\end{frame}

\begin{frame}[fragile]{Reading contents of a file: Word by Word}
\begin{minted}{c++}
std::string filename = "inputword.txt";
std::string word;
std::ifstream inFile;
inFile.open(filename.c_str());
if (!inFile){
        cout<<"Error: Can't open "<<filename<<endl;
}else{
        while(inFile>>word;){
                //do something with word variable
        }
}
inFile.close()
\end{minted}
\end{frame}

\begin{frame}{Reading contents of a file: Word by Word}
	\begin{figure}
		\includegraphics[width=1\textwidth]{readword}
	\end{figure}
\end{frame}


\begin{frame}[fragile]{Reading contents of a file: Numbers}
\begin{minted}{c++}
std::string filename = "inputnum.txt";
int number;
std::ifstream inFile;
inFile.open(filename.c_str());
if (!inFile){
        cout<<"Error: Can't open "<<filename<<endl;
}else{
        while(inFile>>number){
                //do something with number variable
        }
}
inFile.close();
\end{minted}
\end{frame}

\begin{frame}{Reading contents of a file: Numbers}
	\begin{figure}
		\includegraphics[width=1\textwidth]{readnumber}
	\end{figure}
\end{frame}

\begin{frame}[fragile]{Reading in data: Table}
\begin{minted}{c++}
std::string filename = "inputtab.txt";
std::string name;
int age;
std::ifstream inFile;
inFile.open(filename.c_str());
if (!inFile){
    cout<<"Error: Can't open "<<filename<<endl;
}else{
    while(inFile>>name>>age){
       //do something with name, age variables
    }
}
\end{minted}
\end{frame}

\begin{frame}{Reading in data: Table}
	\begin{figure}
		\includegraphics[width=1\textwidth]{readtable}
	\end{figure}
\end{frame}

\begin{frame}[fragile]{Reading in data: line by line}
\begin{minted}{c++}
std::string filename = "input.txt";
std::string line;
std::ifstream inFile;
inFile.open(filename.c_str());
if (!inFile){
    cout<<"Error: Can't open "<<filename<<endl;
}else{
    while(getline(inFile, line)){
       //do something with line variable
    }
}
inFile.close();
\end{minted}
\end{frame}

\begin{frame}{Reading in data: line by line}
	\begin{figure}
		\includegraphics[width=1\textwidth]{readline}
	\end{figure}
\end{frame}

\begin{frame}{Output Stream}
	\begin{figure}
		\includegraphics[width=1\textwidth]{write}
	\end{figure}
\end{frame}

\begin{frame}[fragile]{Writing out data}
\begin{minted}{c++}
std::string filename = "outfile.txt";
std::ofstream outFile;
outFile.open(filename.c_str());
outFile<<"This is text"<<endl;
outFile.close();
\end{minted}
\end{frame}

\begin{frame}{Writing out data: "outfile.txt"}
	\begin{figure}
		\includegraphics[width=.8\textwidth]{out}
	\end{figure}
\end{frame}

\begin{frame}[fragile]{Writing out more data}
\begin{minted}{c++}
std::string filename = "outfile.txt";
std::ofstream outFile;
outFile.open(filename.c_str());
outFile<<"This is text"<<endl;
outFile<<"This is some more text"<<endl;
outFile<<"This is even more text"<<endl;
outFile.close();
\end{minted}
\end{frame}


\begin{frame}{Writing out more data: "outfile.txt"}
	\begin{figure}
		\includegraphics[width=.8\textwidth]{out2}
	\end{figure}
\end{frame}


\begin{frame}[fragile]{Writing out lots more data}
\begin{minted}{c++}
std::string filename = "outfile.txt";
std::ofstream outFile;
outFile.open(filename.c_str());
for (int i=0; i<100; i++){
    outFile<<"This is text"<<endl;
    outFile<<"This is some more text"<<endl;
    outFile<<"This is even more text"<<endl;
}
outFile.close();
\end{minted}
\end{frame}

\begin{frame}{Writing out lots more data: "outfile.txt"}
	\begin{figure}
		\includegraphics[width=.8\textwidth]{out3}
	\end{figure}
\end{frame}

\begin{frame}[fragile]{file stream file modes}
	\begin{center}
	\begin{minted}{c++}
		fileobject.open(filename, mode flag);
	\end{minted}
	\end{center}
\begin{table}
\begin{tabular}{|l|l|p{5cm}|}
	\hline
	\textbf{mode flag} & \textbf{mode} & {access}\\
	\hline
	std::ios::in & input & file open for reading \\
	\hline
	std::ios::out & output & file open for writing\\
	\hline
	std::ios::binary & binary & operations are performed in binary more (not text)\\
	\hline
	std::ios::ate & at end & output position starts at the end of the file \\
	\hline
	std::ios::app & append & ouput operations happen at end of the file (append to existing content)\\
	\hline
	std::ios::trunc & truncate & any contents in file are discarded before opening the file\\
	\hline
\end{tabular}
\end{table}
These flags can be combined with the bitwise OR operator $\mid$
\end{frame}

\begin{frame}{Random Access}
	\begin{block}{Read from a specific location in the file}
			\begin{itemize}
				\item input: inFile.seekg(offset, mode)
				\item output: outFile.seekp(offset, mode)
			\end{itemize}
		offset is the start position as measured in bytes from the file mode (either beginning, current, or end)
	\end{block}
	\begin{block}{Find the current position in the file}
			\begin{itemize}
				\item input: inFile.tellg()
				\item output: outFile.tellp()
			\end{itemize}
	\end{block}
\end{frame}

\begin{frame}[fragile]{String as File: stringstream}
\begin{block}{Convert int to string?}
\begin{minted}{c++}
int clsnum=102;
std::string cls="The class is CS";
cls+=clsnum;
std::cout<<cls<<std::endl;
\end{minted}
\end{block}

\begin{block}{Output}
\begin{minted}{bash}
$ ./strstr.exe
The class is CS f
\end{minted}
\end{block}
\end{frame}

\begin{frame}[fragile]{String as File: stringstream}
\begin{block}{Convert int to string?}
\begin{minted}{c++}
int clsnum=102;
std::string cls="The class is CS";
cls+=clsnum;
std::cout<<cls<<" "<< clsnum<<std::endl;
std::cout<<cls<<std::endl;
\end{minted}
\end{block}

\begin{block}{Output? cls doesn't change!}
\begin{minted}{bash}
$ ./strstr.exe
The class is CSf 102
The class is CSf
\end{minted}
\end{block}
\end{frame}

\begin{frame}[fragile]{String as File: stringstream}
\begin{block}{Convert int to string?}
\begin{minted}{c++}
int clsnum=102;
std::string cls="The class is CS";
cls+=to_string(clsnum);
std::cout<<cls<<" "<<clsnum<<std::endl;
std::cout<<cls<<std::endl;
\end{minted}
\end{block}

\begin{block}{Output? Hard to compile.}
\begin{minted}{bash}
$ g++ -Wall -std=c++11 -o strstr strstr.cpp
strstr.cpp: In function ‘int main()’:
strstr.cpp:12:25: error: ‘to_string’ was not declared 
    in this scope cls+=to_string(clsnum);
\end{minted}
\end{block}
\end{frame}

\begin{frame}[fragile]{String as File: stringstream}
\begin{block}{Convert int to string!}
\begin{minted}{c++}
#include <sstream>
//other header info
int main(){
   int clsnum=102;
   std::stringstream cls; //create a stringstream object
   cls<<"This class is CS"<<clsnum; //write to it
   std::cout<<cls.str()<<std::endl;
}
\end{minted}
\end{block}

\begin{block}{Output}
\begin{minted}{bash}
$ ./strstr.exe
This class is CS102
\end{minted}
\end{block}
\end{frame}

\begin{frame}[fragile]{String Functions}
\begin{block}{String as Vector}
\begin{minted}{c++}
std::string hi = "Hello";
\end{minted}
\begin{table}
	\Huge
	\begin{tabular}{|c|c|c|c|c|c|c|}
	\hline
	H & e & l & l & o & ! & $\backslash$0\\
	\hline
	\end{tabular}
\end{table}
\end{block}
\begin{block}{String Functions}
	\begin{itemize}
		\item hi.length()
		\item hi.size()
		\item hi[ind]
		\item hi.at(ind)
		\item hi.find('a')
	\end{itemize}
\end{block}
\end{frame}


\begin{frame}{Character handling functions}
\begin{tabular}{|l|l|}
\hline
\texttt{isalnum} & Check if is alphanumeric\\ 
 \hline 
\texttt{isalpha} & Check if character is alphabetic\\ 
 \hline 
\texttt{isblank} & Check if character is blank\\ 
 \hline 
\texttt{iscntrl} & Check if character is a control character \\ 
 \hline 
\texttt{isdigit} & Check if character is decimal digit\\ 
 \hline 
\texttt{isgraph} & Check if character has graphical representation\\ 
 \hline 
\texttt{islower} & Check if character is lowercase letter\\ 
 \hline 
\texttt{isprint} & Check if character is printable\\ 
 \hline 
\texttt{ispunct} & Check if character is a punctuation character\\ 
 \hline 
\texttt{isspace} & Check if character is a white-space\\ 
 \hline 
\texttt{isupper} & Check if character is uppercase letter\\ 
 \hline 
\texttt{isxdigit} & Check if character is hexadecimal digit\\ 
 \hline 
\texttt{tolower} & Convert uppercase letter to lowercase\\ 
 \hline 
\texttt{toupper} & Convert lowercase letter to uppercase\\
\hline
\end{tabular}
\end{frame}

\end{document}

