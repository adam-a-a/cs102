\documentclass[xcolor={dvipsnames}]{beamer}
\usepackage{amsmath,amsfonts,amssymb,pxfonts,eulervm,xspace}
\usepackage{graphicx}
 \usepackage{multimedia}
\usepackage{media9}
\usepackage{tabularx}
\usepackage{minted}
\graphicspath{{./figures/}}
\usetheme{ccnycrest}


\newenvironment{changemargin}[2]{%
\begin{list}{}{%
\setlength{\topsep}{0pt}%
\setlength{\leftmargin}{#1}%
\setlength{\rightmargin}{#2}%
\setlength{\listparindent}{\parindent}%
\setlength{\itemindent}{\parindent}%
\setlength{\parsep}{\parskip}%
}%
\item[]}{\end{list}}

\begin{document}

\title{ CS102: Declarations, Initialization, and Assignment }
\author{Hannah Aizenman}

\begin{frame}
	\titlepage
\end{frame}


\begin{frame}{Memory}

\begin{figure}
	\includegraphics[width=1\textwidth]{memory}
\end{figure}


\begin{block}{}
	\begin{description}
		\item[stack] memory set aside for an execution thread (program)
		\item[heap] memory set aside for allocating on the fly
	\end{description}
\end{block}

\end{frame}




\begin{frame}{Declaration Statement}

	\begin{block}{How they're written:}
		\begin{center}
			\textcolor{DarkOrchid}{\textbf{type}} \textit{name};\\
			\pause
			\textcolor{DarkOrchid}{\textbf{type}} \textit{name1}, \textit{name2}, \textit{name3}, ...;
		\end{center}
	\end{block}
\end{frame}

\begin{frame}{Declaration Statement: Type}

\begin{figure}
	\includegraphics[width=1\textwidth]{dtype}
\end{figure}

\begin{block}{}
\begin{itemize}
	\item \textbf{type} allocates space in memory
	\item the primative types are: int, double, float, char, bool
	\item storage size can be specificed using: 
		\begin{itemize}
			\item long, short
			\item unsigned, signed
		\end{itemize}
\end{itemize}
\end{block}

\end{frame}

\begin{frame}{Declaration Statement: Variable Name}

\begin{figure}
	\includegraphics[width=1\textwidth]{label}
\end{figure}

\begin{block}{}
	\begin{itemize}
	 \item \textbf{name} labels space in memory
	\item \textbf{name} must follow the following rules:
		\begin{itemize}	
			\item First character has to be a letter or underscore
			\item Must be composed of letters, digits, and underscores
			\item Can't be part of the language	
		\end{itemize}
	\end{itemize}
\end{block}
\end{frame}

\begin{frame}{Initialization Statement: Value}

\begin{figure}
	\includegraphics[width=1\textwidth]{init}
\end{figure}	

\begin{block}{}
	\begin{itemize}
		\item stores value in location identified by label
		\item value stored as binary number
	\end{itemize}
\end{block}
\end{frame}

\begin{frame}{Assignment Statement: Change}
\begin{figure}
	\includegraphics[width=1\textwidth]{change}
\end{figure}
\begin{block}{}
	\begin{itemize}
		\item stores value in location identified by label
		\item overwrites original value (if any)
	\end{itemize}
\end{block}
	
\end{frame}

\begin{frame}{Assignment Statement: Copy}
\begin{figure}
	\includegraphics[width=1\textwidth]{copy}
\end{figure}
\end{frame}

\begin{frame}{Assignment Statement: Multiple Labels}
\begin{figure}
	\includegraphics[width=1\textwidth]{ptr}
\end{figure}

\begin{block}{}
	\begin{description}
		\item[address operator \textbf{\&}] the address of the variable
		\item[indirection operator \textbf{*}] the value stored at that address		
	\end{description}
\end{block}
\end{frame}

\begin{frame}{Assignment Statement: cin}
\begin{figure}
	\includegraphics[width=1\textwidth]{cin}
\end{figure}

\begin{block}{}
	\begin{itemize}
		\item \textbf{\textgreater\textgreater}  is the stream operator
		\item part of the iostream library (\textbf{\#include \textless iostream\textgreater})
		\item in std namespace (\textbf{using namespace std;})
	\end{itemize}
\end{block}
\end{frame}

\begin{frame}{Assignment Statement: multiple cin}
\begin{figure}
	\includegraphics[width=1\textwidth]{mcin}
\end{figure}

\begin{block}{}
	\begin{itemize}
		\item can have unlimited vars after \textgreater\textgreater
	\end{itemize}
\end{block}
\end{frame}

\begin{frame}{Assignment Statement: Expressions}
\begin{figure}
	\includegraphics[width=1\textwidth]{expr}
\end{figure}

\begin{block}{Note: PEMDAS precedence}
	\begin{tabularx}{\textwidth}{X X X X}
		\textbf{Operator} & \textbf{What it does} & \textbf{Statement} & \textbf{Stored in x}\\
		+ & addition & x = x + 10; & 10 \\
		- & subtraction & x = x - 4 & 6 \\
		*& multiplication & x = x* 3; &  18 \\
		/ & division & x = x/2; & 9 \\
		\% & modulo (remainder) &  x = x\%2; & 1 \\  
	\end{tabularx}
\end{block}
\end{frame}

\begin{frame}[fragile]{Accumulators}
\begin{block}{Operators}
\huge
\begin{minted}{c++}	
	+=, -=, *=, /=, %=
\end{minted}

\end{block}
	
\begin{block}{How to use an accumulator:}
\begin{minted}{c++}
int sum = 10;
sum+=10; //equivalent to sum = sum + 10;
sum/=5; // equivalent to sum = sum /5;
\end{minted}
	\end{block}
\end{frame}

\begin{frame}[fragile]{Counters}
	\center
\begin{block}{prefix}
	%\center
	\begin{tabularx}{\textwidth}{| X | X |}
		\hline
		k=++n;& n = n+1; \\
			   & k = n;\\
		\hline
		k=$\text{-}\text{-}$n; &  n = n-1;\\
			&  k = n;\\
		\hline
	\end{tabularx}	
\end{block}
\begin{block}{postfix}
	%\center
	\begin{tabularx}{\textwidth}{| X | X |}
		\hline
		k=n++; & k = n; \\
			   & n = n + 1;\\
		\hline
		k=n$\text{-}\text{-}$; & k = n;\\
					     &  n = n-1;\\
		\hline
	\end{tabularx}
\end{block}
\end{frame}

\begin{frame}[fragile]{String Object}
		\huge
		\begin{minted}{c++}
		string cn = "CS102";
		string s1  = "K";
		\end{minted}
	\normalsize 

	\begin{tabular}{| l| l | l | c |}
		\hline
		 \textbf{Method}& \textbf{Description} & \textbf{Example} & \textbf{Output}\\

		\hline
		\texttt{int length()} & length & \texttt{cn.length()} & 5\\
		\texttt{int size()} &  & \texttt{cn.size()} & \\
		\hline
		\texttt{int at(int ind)} & character & \texttt{cn.at(3)} & '0'\\
		\texttt{[ind]}&  at location i & \texttt{cn[3]}  &\\

		\hline
		\texttt{int find(str)} & where str & \texttt{cn.find("S1")} & 1\\
					& first occurs& &\\
		\hline
		\texttt{+} & adds strings & \texttt{cn + " "+ s1} & "CS102 K"\\
		\hline
	\end{tabular}
\end{frame}

\begin{frame}{String Input and Output}
	\begin{tabularx}{\textwidth}{| l | X |}
		\hline
		\textbf{Object or Function} & \textbf{What it does}\\
		\hline
		\texttt{cout<<Obj;}& outputs Obj to screen\\
					      & Obj can be of any type\\
		\hline
		\texttt{cin>>Obj;} & stores keyboard input in Obj \\
					    & Obj can be of any type\\
				 	    & stops reading string input when white space is encountered\\
		\hline
		\texttt{getline(cin, Obj);} & stores keyboard input Obj\\
							& Obj must be a string\\
							    & stops reading input when\\
							    & newline is encountered\\
							   
		\hline
	\end{tabularx}
\end{frame}

\end{document}

